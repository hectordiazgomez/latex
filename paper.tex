\documentclass[twocolumn]{article}
\usepackage[utf8]{inputenc}
\usepackage[T1]{fontenc}
\usepackage{lmodern}
\usepackage{textcomp}
\usepackage{lastpage}
\usepackage[spanish]{babel}
\usepackage{csquotes}
\usepackage[style=apa, backend=biber]{biblatex}
\addbibresource{referencias.bib}

\title{Impacto de la Inteligencia Artificial Generativa en el Mercado Laboral de Programadores Freelance}
\author{}
\date{}

\begin{document}

\maketitle

\section*{Introducción}
La introducción de herramientas de inteligencia artificial generativa, como ChatGPT, está transformando el mercado laboral. Particularmente, en el ámbito del desarrollo de software, donde la innovación tecnológica es constante, la influencia de estas herramientas en el desempeño y los ingresos de los programadores \emph{freelance} se espera que sea significativa. Así, este estudio se centra en evaluar empíricamente cómo la introducción y popularización de la IA generativa de uso masivo en noviembre de 2022 ha afectado a los programadores \emph{freelance}, bajo diferentes métricas.

La investigación abordará un vacío crítico en la literatura actual al examinar tres aspectos importantes: los cambios en los ingresos promedio anuales, las variaciones en el tiempo medio de finalización de proyectos, y las alteraciones en las calificaciones promedio de satisfacción del cliente. Se comparará el período de diciembre 2022 a noviembre 2023 con el mismo lapso del año anterior, proporcionando así una visión integral del impacto inmediato y a corto plazo de ChatGPT en este segmento específico del mercado laboral.

Se espera que los resultados de este estudio no solo contribuyan a la comprensión académica del fenómeno, sino que también ofrezcan información valiosa para profesionales, empresas y formuladores de políticas. Al cuantificar el impacto de la IA generativa en métricas concretas de desempeño laboral, se espera proporcionar una base empírica sólida para el desarrollo de estrategias adaptativas y políticas que optimicen el uso de herramientas de IA generativa en el sector del desarrollo de software \emph{freelance}.

\section*{Marco Teórico}
La inteligencia artificial (IA) se ha convertido en una tecnología fundamental que está transformando el mercado laboral, los salarios y la productividad de una manera significativa \parencite{frank2019}. La IA abarca un conjunto de algoritmos y modelos estadísticos, como el aprendizaje automático, que utilizan grandes volúmenes de datos y poder computacional para generar predicciones, recomendaciones y decisiones.

Un área de la inteligencia artificial es la inteligencia artificial generativa, la cual se define como sistemas que generan contenido nuevo, como texto, audio, video y otros tipos de medios, basándose en datos de entrenamiento \parencite{park2023}. La IA generativa se aplica en múltiples áreas, incluyendo la medicina, la informática, el entretenimiento, la creación de contenido y el diseño, entre otros \parencite{kar2023}. En los últimos años, la IA generativa se ha vuelto bastante popular y usada por millones de personas en todo el mundo; eso ha sido posible debido a los Modelos de Lenguaje Grande (LLMs, por sus siglas en inglés), una clase avanzada de modelos de lenguaje, y de los que forman parte los populares ChatGPT, Claude o Gemini, que han demostrado capacidades notables en tareas de procesamiento de lenguaje natural (NLP) y más allá. Estos modelos, que incluyen arquitecturas como los Transformadores, se entrenan en grandes corpus de datos y han revolucionado la forma en que se desarrollan y utilizan los algoritmos de inteligencia artificial. Los LLMs han evolucionado desde modelos estadísticos hasta modelos neuronales, mostrando capacidades avanzadas en la comprensión y generación de lenguaje \parencite{zhao2023}. Estos modelos, al escalarse en tamaño, no solo mejoran en rendimiento, sino que también adquieren habilidades especiales no presentes en modelos más pequeños \parencite{chowdhery2022}. Estas habilidades especiales se asemejan a lo que los seres humanos denominamos "inteligencia".

Los LLMs han demostrado un rendimiento sobresaliente en una variedad de tareas de NLP, incluyendo escritura creativa, traducción, respuesta a preguntas y generación de código. En particular, modelos como GPT-3 o GPT-4 han mostrado una capacidad sorprendente para resolver problemas de analogía sin entrenamiento directo, superando en algunos casos las capacidades humanas. Dadas estas capacidades, los LLMs tienen el potencial de revolucionar múltiples campos, incluyendo la ciencia, la ingeniería, las ciencias sociales y las humanidades, debido a su capacidad para generar lenguaje humano y realizar tareas complejas. Particularmente, la creación de código es una de las tareas en las cuales se enfocan los \emph{benchmarks} que se utilizan para medir y comparar las capacidades de los LLMs.

Una plataforma usada frecuentemente por programadores para realizar trabajos cortos remunerados es Freelancer.com, un mercado en línea que conecta a clientes y \emph{freelancers} para proyectos a corto y largo plazo. Freelancer.com es uno de los mercados de trabajo independiente más grandes del mundo, con más de 76 millones de usuarios (Freelancer, 2023). Específicamente, un trabajador \emph{freelancer} es un individuo que, por lo general, es remunerado por tarea y trabaja por un corto período de tiempo. Es decir, no trabaja para una empresa. Así, pueden realizar varios trabajos para diferentes clientes simultáneamente. Sin embargo, si tienen un contrato para trabajar exclusivamente para un cliente en particular, no pueden tomar otras tareas hasta que hayan completado ese proyecto (The Economic Times).

\section*{Revisión de Literatura}
La rápida evolución de la inteligencia artificial ha transformado la manera en que se abordan tareas laborales y procesos productivos. La mayor parte de la literatura existente se ha enfocado en modelos de IA diseñados para tareas específicas y no necesariamente en IA generativa. Sin embargo, la reciente aparición de modelos de IA generativa, como ChatGPT, ha suscitado interrogantes sobre su efecto potencial en la productividad, la eficiencia y la calidad del trabajo, en especial en campos como el desarrollo de software.

Desde la perspectiva histórica, Keynes (1930) fue uno de los primeros en proponer la idea de "desempleo tecnológico", una consecuencia de la automatización de procesos. Aunque su discusión no aborda específicamente la IA, sus preocupaciones sobre el impacto del progreso tecnológico en el empleo son muy relevantes. Keynes pronosticó que, con el avance tecnológico, la mayoría de las tareas requerirán una menor inversión de tiempo, llevando potencialmente a una semana laboral de 15 horas. Si bien estas predicciones se basaban en desarrollos tecnológicos de la era industrial, reflejan una transición similar a la que se observa hoy con el crecimiento de la IA y la automatización.

Nilsson (1984), por su parte, introduce una visión optimista respecto al impacto de la IA en el trabajo y la producción. Plantea que la IA podría llevar a un aumento significativo de la calidad y eficiencia en la producción de bienes y servicios. Sin embargo, prevé un período de transición complejo en términos laborales, donde la sustitución de trabajo humano por máquinas es inevitable. Propone que, con el tiempo, la economía se adaptará para distribuir la riqueza creada por la IA de manera equitativa.

En estudios más recientes, Brynjolfsson y McAfee (2014) argumentan que el desarrollo de IA y tecnologías asociadas tiene el potencial de aumentar la productividad en múltiples sectores, pero también advierten sobre los riesgos de la polarización del empleo, donde se crea una brecha entre trabajos altamente especializados y aquellos que son fácilmente automatizables. Su metodología consistió en analizar tendencias macroeconómicas relacionadas con la automatización y su impacto en diversos sectores, aunque no profundiza específicamente en el trabajo \emph{freelance} ni en IA generativa.

Desde una perspectiva empírica, Katz y Krueger (2019) estudian cómo la tecnología ha afectado la estructura de empleo en los Estados Unidos, con especial énfasis en el trabajo autónomo y los empleos temporales. Su análisis revela que la creciente demanda de flexibilidad y el acceso a herramientas tecnológicas han fomentado el crecimiento del trabajo \emph{freelance}. Sin embargo, no profundizan en los efectos de la IA generativa y su potencial para alterar el equilibrio entre demanda y oferta de trabajo.

Usando datos de patentes, Damioli et al. (2021) concluyen que las familias de patentes de IA tienen un impacto positivo y significativo en el empleo, lo que indica que la innovación de productos de IA tiene una naturaleza favorable al trabajo. Sin embargo, su estudio es del 2021, previo al boom de la IA generativa, la cual, a diferencia de las tecnologías de automatización anteriores que se centraban en tareas rutinarias, demuestra capacidades de automatizar tareas no rutinarias, incluidas tareas cognitivas y de alta complejidad \parencite{oecd2023}. Esto implica que trabajos previamente considerados como inmunes a la automatización están ahora en riesgo de ser transformados por la IA generativa.

En un contexto similar, Bessen et al. (2020) investigan cómo la adopción de IA puede influir en las disparidades de ingresos, con resultados que muestran que la IA puede aumentar la productividad, pero también intensificar la desigualdad salarial. La metodología del estudio se basa en análisis estadísticos de datos de salarios y adopción tecnológica en diferentes industrias. La investigación sugiere que la adopción de IA puede llevar a una mayor concentración de ingresos en trabajadores altamente calificados, lo que podría influir en los ingresos de programadores \emph{freelance} según su nivel de habilidades y experiencia. Otro factor, cuya relevancia sería destacada dada la IA generativa, son las habilidades blandas. Al respecto, Necula (2023) considera que los futuros ingenieros de software necesitarán no solo competencias en programación, sino también habilidades blandas como la resolución de problemas y la comunicación interpersonal para adaptarse a un entorno de trabajo asistido por IA; así, los profesionales deberán adaptarse a este nuevo panorama para seguir siendo relevantes y efectivos.

En general, dependiendo de la naturaleza del trabajo, se puede afirmar que la IA generativa puede desplazar completamente a los trabajadores o aumentar su productividad. Dada esta situación tan relativa, también se debe considerar la opción de que los requisitos de habilidades de las ocupaciones no permanecen estáticos, sino que cambian con la tecnología en evolución \parencite{autor2003}, lo cual hace aún más difícil estimar el impacto futuro de la IA generativa en determinados tipos de trabajos.

Se puede, sin embargo, con las herramientas económicas existentes, tratar de predecir el efecto final de la IA generativa en el empleo. Para esto, vamos a considerar que hay un efecto sustitución, debido a la automatización, que resulta en la pérdida de trabajos debido a que la IA generativa puede realizarlos, y también hay un efecto productividad, el cual aumentaría la demanda laboral debido a los ahorros de costos y las mejoras de eficiencia que trae consigo el uso de la IA generativa. Estos ahorros pueden generar nuevas oportunidades de empleo en tareas no automatizadas o en áreas donde el trabajo humano y la IA se complementan \parencite{oecd2023}.

Así, si el efecto productividad es más fuerte, hará caer los precios lo suficiente para aumentar la demanda y, así, aumentar el empleo en trabajos aún ocupados total o parcialmente por seres humanos. El balance entre estos efectos es ambiguo y depende de múltiples factores, como la velocidad de adopción de la tecnología, la estructura del mercado laboral y la capacidad de los trabajadores para adaptarse y adquirir nuevas habilidades \parencite{oecd2023}. Los efectos de productividad tienden a predominar cuando la IA se utiliza para crear nuevas tareas y mejorar procesos, en lugar de simplemente reemplazar a la mano de obra humana.

La literatura existente apunta a una serie de tendencias clave en la relación entre IA y empleo. Mientras que los estudios iniciales se centraron en los impactos generales de la automatización (Keynes, 1930; Nilsson, 1984), investigaciones más recientes han explorado la relación específica entre IA y productividad, aunque con un enfoque limitado en la IA generativa y el trabajo \emph{freelance} \parencite{brynjolfsson2014, katz2019}. El contexto actual, marcado por la irrupción de herramientas generativas como ChatGPT, requiere un análisis más detallado sobre cómo estas tecnologías afectan las métricas de productividad, tiempo de finalización de trabajos y calidad percibida, especialmente en segmentos laborales específicos como los programadores \emph{freelance} en plataformas como Freelancer.com.

\printbibliography

\end{document}
